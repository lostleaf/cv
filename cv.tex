% LaTeX Curriculum Vitae Template
%
% Copyright (C) 2004-2009 Jason Blevins <jrblevin@sdf.lonestar.org>
% http://jblevins.org/projects/cv-template/
%
% You may use use this document as a template to create your own CV
% and you may redistribute the source code freely. No attribution is
% required in any resulting documents. I do ask that you please leave
% this notice and the above URL in the source code if you choose to
% redistribute this file.

\documentclass[letterpaper]{article}

\usepackage{xeCJK}
\setCJKmainfont{Kai}

\usepackage{fontspec}
\setmainfont
  [ BoldFont   = HelveticaNeue-Bold.otf,
    ItalicFont = HelveticaNeue-LightItalic.otf ]
  {HelveticaNeue-Light.otf}
\setsansfont{HelveticaNeue-Light.otf}
% \setmonofont{Courier New}

\usepackage{hyperref}
\usepackage{geometry}

% Set your name here
\def\name{Qinglin Li}
% Chinese name here
\def\cname{李青林}

% Replace this with a link to your CV if you like, or set it empty
% (as in \def\footerlink{}) to remove the link in the footer:
\def\footerlink{}

% The following metadata will show up in the PDF properties
\hypersetup{
  colorlinks = true,
  urlcolor = black,
  pdfauthor = {\name},
  pdfkeywords = {computer science, machine learning},
  pdftitle = {\name: Curriculum Vitae},
  pdfsubject = {Curriculum Vitae},
  pdfpagemode = UseNone
}

\geometry{
  body={6.5in, 9.5in},
  left=0.8in,
  top=0.8in
}

% Customize page headers
\pagestyle{myheadings}
\markright{\name}
\thispagestyle{empty}

% Custom section fonts
\usepackage{sectsty}
\sectionfont{\large}

% Other possible font commands include:
% \ttfamily for teletype,
% \sffamily for sans serif,
% \bfseries for bold,
% \scshape for small caps,
% \normalsize, \large, \Large, \LARGE sizes.

% Don't indent paragraphs.
\setlength\parindent{0em}

\usepackage{wasysym}

\begin{document}

\begin{minipage}{0.4\linewidth}
  {\huge \name }
  \vspace{0.1in} \\
  Born on June 1, 1993. Male. \\
  \href{http://www.sjtu.edu.cn/}{Shanghai Jiao Tong University}
\end{minipage}
\begin{minipage}{0.45\linewidth}
  \begin{tabular}{ll}
    Gmail:  & \href{mailto:jack951753@gmail.com}{\tt jack951753@gmail.com} \\
    GitHub: & \href{https://github.com/lostleaf}{\tt http://github.com/lostleaf} \\
    Tel:    & {\tt +86 151-2112-7746}
  \end{tabular}
\end{minipage}

~\\
I'm a student of \textbf{\href{http://acm.sjtu.edu.cn}{ACM Honored Class}} of Shanghai Jiaotong University.
I started programming around 2003 and I'm experienced in various programming languages and techniques.
And I'm interested in \emph{web development} and \emph{software engineering}.
My primary research interest includes \emph{machine learning} and \emph{computer vision}.
\section*{Education}
\begin{itemize}

\item  B.S. in Computer Science.~\quad\qquad~2011.9 - 2015.6~(expected) \\
    \emph{\href{http://acm.sjtu.edu.cn}{ACM Honored Class},
    \href{http://www.sjtu.edu.cn/}{Shanghai Jiao Tong University}.}
\end{itemize}

\section*{Personal Experience}
\begin{itemize}
\item \textbf{Research Intern}, \emph{\href{http://research.microsoft.com/en-us/labs/asia/}{Microsoft Research Asia}} \quad2014.8 - 2015.2~(excepted)\\
    Advisor:~~~~~~~~~~~~~~{Dr. Mu Li}\\
    Research Area:~~~Natural Language Processing.
\item \textbf{Research Assistant}, \emph{\href{http://bcmi.sjtu.edu.cn}{BCMI Lab}}\qquad\qquad\qquad~~2013.7 - present\\
    Advisor:~~~~~~~~~~~~~~{\href{http://bcmi.sjtu.edu.cn/~zhangliqing/}{Prof. Liqing Zhang}}\\
    Research Area:~~~Machine Learning and Computer Vision.
\item \textbf{Teaching Assistant},  \emph{\href{http://acm.sjtu.edu.cn/wiki/Programming_2013}{C++ Programming}}\qquad~ 2013.9 - 2014.1\\
Leader of the TA team.
\item \textbf{Teaching Assistant},  \emph{C++ Programming}\qquad~ 2012.9 - 2013.1\\
\end{itemize}
\section*{Selected Projects \normalsize{\tt(\href{https://github.com/lostleaf?tab=repositories}{more details})}}
\begin{itemize}
\item \textbf{Crowd Density Estimation in Video}:
a college student innovation project, which estimate the crowd density with machine learning techniques in videos. The final gaol is to enhance the performance in high density and high occlusion scenes, such as surveillance video of subway stations.
\item \textbf{\href{http://acm.sjtu.edu.cn/ricsrt/}{Ranking Tool Research Capacity in CS}}:
a project for \emph{\href{http://www.cs.cornell.edu/jeh/}{Prof. John Hopcroft}}'s seminar, which visualizes and compares the CS research productivity, quality and impact of countries and institutions.
\item \textbf{\href{https://github.com/lostleaf/fatworm}{Fatworm Database}}:
a complete relational database system built from scratch in java. It contains a implementation of SQL parser, SQL engine and JDBC interface.
control, database manager
\item \textbf{\href{https://github.com/lostleaf/compiler}{C Language Compiler}}:
a project written in Java, which supports most features of C Language and targets the MIPS architecture. It contains a full implementation of several optimization.
\item \textbf{\href{https://github.com/lostleaf/nachos}{Nachos Operating System}}: 
a project written in Java, including threading and multiprogramming, cache and virtual memory, and a self-designed file system.
\item \textbf{\href{https://github.com/lostleaf/cpu}{Simulated CPU}}:
a project written in verilog, including a full implementation of tomasulo algorithm with buffers. It is able to execute a turing complete subset of MIPS assembly language.
\item \textbf{\href{http://acm.sjtu.edu.cn/OnlineJudge/}{SJTU Online Judge}}: (Lead by \emph{\href{http://xiao-jia.com/}{Xiao Jia}}) a publicly available online system written in PHP for testing programs in programming contests, which is used for auto-grading homeworks and exams for several courses and leverages open source softwares such as Linux, nginx, Varnish, MySQL, Redis, memcached, etc.
\end{itemize}

\section*{Awards}
\begin{itemize}
\item  \textbf{Honorable Mention(Second Prize)} in Mathematical Contest in Modeling (MCM), 2014.
\item  \textbf{Third Prize} in China Undergraduate Mathematical Contest in Modeling (CUMCM), 2013. 
\item  \textbf{Silver Medal(Second Prize)} in China's National Olympiad in Informatics (NOI), 2010. 
\item  \textbf{First Prize} in China's National Olympiad in Informatics in Provinces (NOIP), 2009. 
\end{itemize}

\section*{Skills}
\begin{itemize}
\item Machine learning related: Python, Matlab, C/C++.
\item Web development related: Ruby, Rails, HTML, javascript, CSS, PHP.
\item Desktop application development related: Java.
\end{itemize}
\end{document}
