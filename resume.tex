%%%%%%%%%%%%%%%%%%%%%%%%%%%%%%%%%%%%%%%%%
% Medium Length Professional CV
% LaTeX Template
% Version 2.0 (8/5/13)
%
% This template has been downloaded from:
% http://www.LaTeXTemplates.com
%
% Original author:
% Trey Hunner (http://www.treyhunner.com/)
%
% Important note:
% This template requires the resume.cls file to be in the same directory as the
% .tex file. The resume.cls file provides the resume style used for structuring the
% document.
%
%%%%%%%%%%%%%%%%%%%%%%%%%%%%%%%%%%%%%%%%%

%----------------------------------------------------------------------------------------
%	PACKAGES AND OTHER DOCUMENT CONFIGURATIONS
%----------------------------------------------------------------------------------------

\documentclass{resume} % Use the custom resume.cls style

\usepackage[left=0.75in,top=0.6in,right=0.75in,bottom=0.6in]{geometry} % Document margins
\usepackage{hyperref}
\usepackage{enumerate}
\name{Qinglin Li} % Your name
\address{2214 Maple Ave 204 APT, Evanston, IL, 60201} % Your address
\address{(847) 246-2457\\ \href{mailto:qinglinli2015@u.northwestern.edu}{qinglinli2015@u.northwestern.edu}}


\begin{document}
\begin{rSection}{Objective}
I'm looking for a 2016 summer internship in software development or data science related area.
\end{rSection}
%----------------------------------------------------------------------------------------
%	EDUCATION SECTION
%----------------------------------------------------------------------------------------
\vspace{10pt}

\begin{rSection}{Education}
\begin{rSubsection}{Northwestern University}{SEP 2015 - DEC 2016 (Expected)}{}\\
M.S. in Computer Science\\
Coursework: Machine Learning, Biometrics, Data Science
\end{rSubsection}
\vspace{5pt}
\begin{rSubsection}{Shanghai Jiao Tong University, China}{SEP 2011 - JUN 2015}{\quad GPA: {86.8/100}}\\
B.S. in Computer Science \& Engineering\\
Coursework: Statistical Learning, Scientific Computing, Algorithms,  Operating System, Computer Network, Database, Computer Architecture, etc.
\end{rSubsection}
\end{rSection}
%----------------------------------------------------------------------------------------
%	TECHNICAL STRENGTHS SECTION
%----------------------------------------------------------------------------------------
\vspace{10pt}

\begin{rSection}{Skills}

\begin{tabular}{ @{} >{\bfseries}l @{\hspace{4ex}} l }
Programming Languages & C/C++, Java, Ruby, Python, MATLAB\\
Application Development & HTML, CSS, Bootstrap, Ruby on Rails, Java Swing\\
Machine Learning \& Vision & scikit-learn, scikit-image, OpenCV\\
Tools \& Others & Vim, \LaTeX{}, Linux/Mac OS X, MySQL, Git
\end{tabular}

\end{rSection}
%----------------------------------------------------------------------------------------
%	WORK EXPERIENCE SECTION
%----------------------------------------------------------------------------------------
\vspace{10pt}

\begin{rSection}{Experience}
\begin{rSubsection}{Research Intern}{AUG 2014 - FEB 2015}{Microsoft Research Asia, Beijing}

Team: Natural Language Computing, Mentor: Dr. Mu Li
\begin{itemize}
\item Collected text data from Internet and analyzed the English-Chinese relationship with rules, using Python and Mechanize.
\item Implemented a graphical model to solve entities translation problem between English and Chinese knowledge bases by utilizing the relations between entities. Outperformed traditional machine translation algorithms. Using Python, NumPy and scikit-learn.
\end{itemize}
\end{rSubsection}
\vspace{5pt}
\begin{rSubsection}{Research Student}{JUL 2013 - AUG 2014}{BCMI Lab, Shanghai Jiao Tong University}

Research Area: Computer Vision, Mentor: Prof. Liqing Zhang
\begin{itemize}
\item Worked on mining rich low-level computer vision features and testing different regression models to solve crowd density estimation problem, using Python, scikit-learn and OpenCV.
\end{itemize}
\end{rSubsection}
\end{rSection}
%----------------------------------------------------------------------------------------
%	COURSE PROJECT EXPERIENCE SECTION
%----------------------------------------------------------------------------------------
\vspace{10pt}

\begin{rSection}{Course Projects}{\begin{bf}GitHub\end{bf}: \href{https://github.com/lostleaf?tab=repositories}{https://github.com/lostleaf}}

\begin{rSubsection}{Database System Lab}{FALL 2014}{Shanghai Jiao Tong University}
\begin{itemize}
\item Developed a relational database system supporting JDBC interface including SQL parser, query engine and file storage in Java.
\end{itemize}
\end{rSubsection}
\vspace{3pt}
\begin{rSubsection}{Nachos Operating System Lab}{FALL 2013}{Shanghai Jiao Tong University}
\begin{itemize}
\item Implemented threading and multiprogramming, virtual memory, and a self-designed file system within the Nachos operating system in Java.
\end{itemize}
\end{rSubsection}
\vspace{3pt}
\begin{rSubsection}{Compiler Lab}{SPRING 2013}{Shanghai Jiao Tong University}
\begin{itemize}
\item Developed a compiler supporting most features of C Language and targeting the MIPS architecture including parser, syntax checking and linear scan algorithm for register allocation.
\end{itemize}
\end{rSubsection}
\vspace{3pt}
\begin{rSubsection}{Computer Architecture Lab}{SPRING 2013}{Shanghai Jiao Tong University}
\begin{itemize}
\item Simulated a CPU with Tomasula algorithm for dynamic instruction scheduling in Verilog.
\end{itemize}
\end{rSubsection}

\end{rSection}
\vspace{10pt}
%----------------------------------------------------------------------------------------
%	AWARDS SECTION
%----------------------------------------------------------------------------------------




%----------------------------------------------------------------------------------------

\end{document}